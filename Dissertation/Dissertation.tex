\pdfoutput=1

\documentclass{4thYearProject}

\makeatletter\@openrightfalse\makeatother
\usepackage{graphicx}
\usepackage{hyperref}
\usepackage{float}
\usepackage{longtable}
\usepackage{listings}
\usepackage{color}
\usepackage{pdfpages}

\definecolor{dkgreen}{rgb}{0,0.6,0}
\definecolor{gray}{rgb}{0.5,0.5,0.5}
\definecolor{mauve}{rgb}{0.58,0,0.82}

\lstset{frame=tb,
  language=SQL,
  aboveskip=3mm,
  belowskip=3mm,
  showstringspaces=false,
  columns=flexible,
  basicstyle={\small\ttfamily},
  numbers=none,
  numberstyle=\tiny\color{gray},
  keywordstyle=\color{blue},
  commentstyle=\color{dkgreen},
  stringstyle=\color{mauve},
  breaklines=true,
  breakatwhitespace=true,
  tabsize=3,
  belowskip=1em
}

\graphicspath{ {resources/images/} }

\begin{document}
\title{Microissues IntelliJ Plugin !NAME TO BE CONSIDERED!}
\author{Alex Leet}
\date{2016/2017}
\maketitle

\begin{abstract}
Abstract here
\end{abstract}

\educationalconsent
%
%NOTE: if you include the educationalconsent (above) and your project is graded an A then
%      it may be entered in the CS Hall of Fame
%
\tableofcontents
%==============================================================================

\chapter{Introduction}
\pagenumbering{arabic}

\section{Project Context}

Testing citation \cite{microissues}


\section{Motivation}


\section{Objectives}


\section{Achievements}


\section{Dissertation Structure}

Go over the structure of the dissertation.

\begin{table}[H]
\caption{Dissertation Structure}
\centering
\def\arraystretch{1.5}
\begin{tabular}{p{3cm}p{12cm}}
\hline
Chapter & Content \\
\hline
2. Background & A short description of the previous work related to the project. \\
3. Requirements & Discusses the gathering of the requirements, followed by a list of all requirements. \\
4. Design and Implementation & Describes the architecture of the plugin. Details the design decisions taken. Also contains a list of final design features.\\
5. Evaluation & Describes the test process and the user evaluation carried out, their results and a discussion of the results. \\
6. Conclusion: & Discusses the outcome of the project, future work possibilities and learning outcomes.  \\
\hline
\end{tabular}
\label{table:reportStructure}
\end{table}


\chapter{Background}

Text on what this chapter talks about

\section{Issue Tracking}

A short section on issue-tracking systems and what they are.

\section{Related Work}

Related plugins that have been created.

\section{Research on Issue Tracking and Usage of Issue-tracking Systems}

\chapter{Requirements}

\section{Requirements Elicitation and Gathering}


\section{Functional Requirements}

\newpage
\section{Non-Functional Requirements}

\chapter{Design and Implementation}

Discuss design choices, justifications and implementation.

\section{Iterations Overview}

Overview of the development iterations.

\section{User Interface}

\chapter{Evaluation}

Describe the aims of this chapter

\section{Unit Testing}
\section{Acceptance Testing}


\chapter{Conclusion}

\section{Project Outcome}

Outline the project outcome. 

\section{Learning Outcomes}


%%%%%%%%%%%%%%%%%%%%
%   BIBLIOGRAPHY   %
%%%%%%%%%%%%%%%%%%%%
\bibliographystyle{ieeetr}
\bibliography{resources/bibliography}

%%%%%%%%%%%%%%%%
%              %
%  APPENDICES  %
%              %
%%%%%%%%%%%%%%%%

\begin{appendices}
\end{appendices}

\end{document}
